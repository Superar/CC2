\documentclass[a4paper,10pt]{article}

    \usepackage{longtable}
    \usepackage{colortbl}%
    \newcommand{\myrowcolour}{\rowcolor[gray]{0.925}}
    \usepackage{booktabs}
	\usepackage{pgfgantt}
	\usepackage{pdflscape}
	\usepackage[utf8]{inputenc}
	\usepackage[margin=.25in]{geometry}
	
	\pagenumbering{gobble}
    
    \definecolor{corAtividade1}{RGB}{0, 0, 255}
    \definecolor{corAtividade2}{RGB}{255, 0, 0}
    
    \begin{document}

\begin{longtable}{p{0.97\textwidth}}
	\toprule                                                 %
	\myrowcolour                                             %
	\bfseries Atividade 1: Atividade1                        \\
	\midrule
	Esta é a descrição da atividade 1.
	\\

	\midrule
	\myrowcolour                                             %
	\bfseries Atividade 2: Atividade2                        \\
	\midrule
	Este texto descreve o que será realizado na atividade 2
	\\

	\midrule
	\myrowcolour                                             %
	\bfseries Atividade 3: Atividade3                        \\
	\midrule
	Texto da atividade 3
	\\

	\bottomrule
\end{longtable}

\newpage

\begin{landscape}
	\begin{center}
		\begin{ganttchart}[
				hgrid,
				vgrid,
				time slot format=little-endian,
				inline,
				compress calendar
			]{01.01.2016}{31.05.2016}
			\gantttitlecalendar{year, month} \\
			\ganttbar[bar/.append style={fill=corAtividade1, inner sep=0pt}]{Atividade1}{01.02.2016}{31.03.2016} \\
			\ganttbar[bar/.append style={fill=corAtividade2, inner sep=0pt}]{Atividade2}{01.01.2016}{31.05.2016} \\
			\ganttbar{Atividade3}{01.01.2016}{30.04.2016}
			\ganttlink{elem0}{elem1}
		\end{ganttchart}
	\end{center}
\end{landscape}

\end{document}