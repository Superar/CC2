\documentclass[a4paper,10pt]{article}

\usepackage{longtable}
\usepackage{colortbl}
\newcommand{\myrowcolour}{\rowcolor[gray]{0.925}}
\usepackage{booktabs}
\usepackage{pgfgantt}
\usepackage{pdflscape}
\usepackage[utf8]{inputenc}
\usepackage[margin=.25in]{geometry}

\pagenumbering{gobble}

\begin{document}

\begin{longtable}{p{0.97\textwidth}}
	\toprule
	\myrowcolour
	\bfseries Atividade 1: EscreverGramatica                                                                 \\
	\midrule
	Escrever a gramática de acordo com as especificações no documento PDF disponibilizado pelo professor.
	\\

	\midrule
	\myrowcolour
	\bfseries Atividade 2: TratarErrosLexicos                                                                \\
	\midrule
	Fazer o tratamento dos erros léxicos de acordo com os exemplos disponibilizados pelo professor.
	\\

	\midrule
	\myrowcolour
	\bfseries Atividade 3: TratarErrosSintaticos                                                             \\
	\midrule
	Fazer o tratamento dos erros sintáticos de acordo com os exemplos disponibilizados pelo professor.
	\\

	\midrule
	\myrowcolour
	\bfseries Atividade 4: TratarErrosSemanticos                                                             \\
	\midrule
	Fazer o tratamento de erros semânticos de acordo com os exemplos disponibilizados pelo professor.
	\\

	\midrule
	\myrowcolour
	\bfseries Atividade 5: GeracaoDeCodigo                                                                   \\
	\midrule
	Implementar a geração de código em C a partir de uma entrada em LA.
    \\
    
    \midrule
	\myrowcolour
	\bfseries Atividade 6: Documentacao                                                                   \\
	\midrule
    Revisar o código, comentar trechos importantes e descrever o processo de compilação e execução do compilador em um arquivo README. O projeto do compilador juntamente com os documentos gerados devem ser
	enviados via AVA.
    \\

	\bottomrule
\end{longtable}

\newpage

\begin{landscape}
	\begin{center}
		\begin{ganttchart}[
				vgrid,
				time slot format=little-endian
			]{09/08/2018}{19/09/2018}
			\gantttitlecalendar{year, month, day} \\

            \ganttmilestone{Atividade 1}{09/08/2018} \\
			\ganttbar{Atividade 2}{13/08/2018}{15/08/2018} \\
            \ganttbar{Atividade 3}{13/08/2018}{15/08/2018} \\
            \ganttbar{Atividade 4}{16/08/2018}{02/09/2018} \\
            \ganttbar{Atividade 5}{07/09/2018}{17/09/2018} \\
			\ganttmilestone{Atividade 6}{19/09/2018}
			\ganttlink{elem0}{elem1}
			\ganttlink{elem1}{elem2}
			\ganttlink{elem2}{elem3}
			\ganttlink{elem3}{elem4}
		\end{ganttchart}
	\end{center}
\end{landscape}

\end{document}