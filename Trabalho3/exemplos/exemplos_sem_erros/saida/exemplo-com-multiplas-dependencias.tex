\documentclass[a4paper,10pt]{article}
\usepackage{longtable}
\usepackage{colortbl}
\newcommand{\myrowcolour}{\rowcolor[gray]{0.925}}
\usepackage{booktabs}
\usepackage{pgfgantt}
\usepackage{pdflscape}
\usepackage[utf8]{inputenc}
\usepackage[margin=.25in]{geometry}
\pagenumbering{gobble}
\begin{document}
\definecolor{ExemploMultiplasDependencias}{HTML}{FFFFFF}
\definecolor{ExemploMultiplasDependencias-Atividade1}{RGB}{0,0,255}
\begin{longtable}{p{0.97\textwidth}}
\toprule
\myrowcolour
\bfseries Atividade 1: Atividade1 \\
\midrule
Esta é a descrição da atividade 1.
\\
\\
Datas de execu\c{c}\~{a}o:\\
01/02/2016 - 01/03/2016\\
01/07/2016 - 01/09/2016\\
\midrule
\myrowcolour
\bfseries Atividade 2: Atividade2 \\
\midrule
Este texto descreve o que será realizado
            na atividade 2
\\
\\
Datas de execu\c{c}\~{a}o:\\
01.06.2016 - 01.10.2016\\
\bottomrule
\end{longtable}
\newpage
\begin{landscape}
\begin{center}
\begin{ganttchart}[hgrid,vgrid,compress calendar,time slot format=little-endian,bar/.append style={fill=ExemploMultiplasDependencias, inner sep=0pt},milestone/.append style={fill=ExemploMultiplasDependencias, inner sep=0pt},bar height=0.5]{01/02/2016}{01/10/2016}
\gantttitlecalendar{year, month}\\
\ganttbar[name=Atividade1.1, bar/.append style={fill=ExemploMultiplasDependencias-Atividade1, inner sep=0pt}]{Atividade 1}{01/02/2016}{01/03/2016}
\ganttbar[name=Atividade1.2, bar/.append style={fill=ExemploMultiplasDependencias-Atividade1, inner sep=0pt}]{}{01/07/2016}{01/09/2016}\\
\ganttbar[name=Atividade2.1]{Atividade 2}{01/06/2016}{01/10/2016}\\
\ganttlink{Atividade1.1}{Atividade2.1}
\end{ganttchart}
\end{center}
\end{landscape}
\end{document}
