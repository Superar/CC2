\documentclass[a4paper,10pt]{article}
\usepackage{longtable}
\usepackage{colortbl}
\newcommand{\myrowcolour}{\rowcolor[gray]{0.925}}
\usepackage{booktabs}
\usepackage{pgfgantt}
\usepackage{pdflscape}
\usepackage[utf8]{inputenc}
\usepackage[margin=.25in]{geometry}
\pagenumbering{gobble}
\begin{document}
\definecolor{Trabalho1}{HTML}{FFFFFF}
\begin{longtable}{p{0.97\textwidth}}
\toprule
\myrowcolour
\bfseries Atividade 1: EscreverGramatica \\
\midrule
Escrever a gramática de acordo com as especificações
            no documento PDF disponibilizado pelo professor.
\\
\\
Datas de execu\c{c}\~{a}o:\\
09/08/2018\\
\midrule
\myrowcolour
\bfseries Atividade 2: TratarErrosLexicos \\
\midrule
Fazer o tratamento dos erros léxicos de acordo com os
            exemplos disponibilizados pelo professor.
\\
\\
Datas de execu\c{c}\~{a}o:\\
13/08/2018 - 15/08/2018\\
\midrule
\myrowcolour
\bfseries Atividade 3: TratarErrosSintaticos \\
\midrule
Fazer o tratamento dos erros sintáticos de acordo com os
            exemplos disponibilizados pelo professor.
\\
\\
Datas de execu\c{c}\~{a}o:\\
15/08/2018\\
\midrule
\myrowcolour
\bfseries Atividade 4: TratarErrosSemanticos \\
\midrule
Fazer o tratamento de erros semânticos de acordo com os
            exemplos disponibilizados pelo professor.
\\
\\
Datas de execu\c{c}\~{a}o:\\
16/08/2018 - 02/09/2018\\
\midrule
\myrowcolour
\bfseries Atividade 5: GeracaoDeCodigo \\
\midrule
Implementar a geração de código em C a partir de uma entrada
            em LA.
\\
\\
Datas de execu\c{c}\~{a}o:\\
07/09/2018 - 17/09/2018\\
\midrule
\myrowcolour
\bfseries Atividade 6: Documentacao \\
\midrule
Revisar o código, comentar trechos importantes e descrever o
            processo de compilação e execução do compilador em um arquivo README.
            O projeto do compilador juntamente com os documentos gerados devem ser
            enviados via AVA.
\\
\\
Datas de execu\c{c}\~{a}o:\\
19/09/2018\\
\bottomrule
\end{longtable}
\newpage
\begin{landscape}
\begin{center}
\begin{ganttchart}[vgrid,time slot format=little-endian,bar/.append style={fill=Trabalho1, inner sep=0pt},milestone/.append style={fill=Trabalho1, inner sep=0pt},bar height=0.5]{09/08/2018}{19/09/2018}
\gantttitlecalendar{year, month, day}\\
\ganttmilestone[name=EscreverGramatica.1]{Atividade 1}{09/08/2018}\\
\ganttbar[name=TratarErrosLexicos.1]{Atividade 2}{13/08/2018}{15/08/2018}\\
\ganttmilestone[name=TratarErrosSintaticos.1]{Atividade 3}{15/08/2018}\\
\ganttbar[name=TratarErrosSemanticos.1]{Atividade 4}{16/08/2018}{02/09/2018}\\
\ganttbar[name=GeracaoDeCodigo.1]{Atividade 5}{07/09/2018}{17/09/2018}\\
\ganttmilestone[name=Documentacao.1]{Atividade 6}{19/09/2018}\\
\ganttlink{EscreverGramatica.1}{TratarErrosLexicos.1}
\ganttlink{TratarErrosLexicos.1}{TratarErrosSintaticos.1}
\ganttlink{TratarErrosSintaticos.1}{TratarErrosSemanticos.1}
\ganttlink{TratarErrosSemanticos.1}{GeracaoDeCodigo.1}
\end{ganttchart}
\end{center}
\end{landscape}
\end{document}
